%%%
%%% cns95.tex (1.2)
%%%
%%% This is a macro package to format papers for the CNS*95 proceedings.
%%% Original versions are available via anonymous ftp at
%%% ftp.bbb.caltech.edu:/pub/cns95/cns95.tex.
%%%
%%% For usage, see the (commented-out) sample paper at the end of this file.
%%%
%%% J. S. Pezaris
%%% Caltech 216-76
%%% Pasadena, CA 91125
%%%
%%% pz@caltech.edu, 7/95.


%%% History
%%%
%%% 09/07/95 1.0 Initial Release
%%%
%%% 09/11/95 1.1 Fixed \author, \authors bug as pointed out by Markus Diesmann;
%%%              made similar changes to \email, \emails.  Also changed inter-
%%%              major section skip from 1cm to 1cm plus 0.1cm minus 0.1cm.
%%%
%%% 09/14/95 1.2 Fixed problem with LaTeX bibliography usage which depended
%%%	         on the native \section being around.  This clashed with the
%%%	         \section defined here.  Thus \section was renamed to
%%%              \minorsection, and under LaTeX, calls to \section* are
%%%              routed to \majorsection, and calls to \section are routed
%%%              to \minorsection.  Under TeX, things work as before.  This
%%%              is not an ideal solution, but provides consistency in the
%%%              interface until next year when a better package can be
%%%              released. 


%%% Are we running under TeX or LaTeX?
%%%

\newif\ifTeX
\newif\ifLaTeX
\ifx\newcommand\undefined
  %% TeX!
  \TeXtrue
  \LaTeXfalse
\else
  %% LaTeX!
  \TeXfalse
  \LaTeXtrue
\fi


%%% Set the basic parameters
%%%

\ifTeX
  \parindent     = 0pt
  \parskip       = \medskipamount
  \topskip       = 0truein
  \magnification = \magstep0
  \hoffset       = 0.875truein
  \voffset       = 0.50truein
  \hsize         = 4.75truein
  \vsize         = 7.25truein
  \leftskip      = 0pt
  \nopagenumbers
\else
  \parindent     = 0pt
  \parskip       = \medskipamount
  \topskip       = 0.0truein
  \topmargin     = 0.5truein
  \headheight    = 0.0truein
  \headsep       = 0.0truein
  \textwidth     = 4.75truein
  \textheight    = 7.25truein
  \pagestyle{empty}
  \flushbottom
  \def\it{\em}
\fi


%%% Define the various fonts used
%%%

\font\thetitlefont        = cmbx12 at 17pt
\font\theauthorfont       = cmbx12 at 15pt
\font\theextrafont        = cmbx12	     % used only for big LaTeX logo
\font\theemailfont        = cmtt12
\font\theinstitutionfont  = cmti12
\font\themajorsectionfont = cmcsc10 at 12pt
\font\thesectionfont      = cmti10


%%% And little macros to invoke them, insuring that the
%%% appropriate \baselineskip is set

\def\titlefont{\thetitlefont\baselineskip=19pt}
\def\authorfont{\theauthorfont\baselineskip=17pt}
\def\emailfont{\theemailfont\baselineskip=14pt}
\def\institutionfont{\theinstitutionfont\baselineskip=14pt}
\def\majorsectionfont{\themajorsectionfont\baselineskip=14pt}
\def\sectionfont{\thesectionfont\baselineskip=12pt}


%%% \titleskip
%%%
%%% Creates a blank area between items in the tite.

\def\titleskip{\vskip 19pt}


%%% \raggedleft
%%%
%%% An equivalent to \raggedright: creates right (but not left)
%%% justified text.

\def\raggedleft{\leftskip 0pt plus 1fill
  \spaceskip=0.333em \xspaceskip=0.5em 
  \rightskip 0pt
  \let\par=\break}


%%% \beginlinemode and \endmode
%%%
%%% This wierd stuff is used to insure line breaks in the source are
%%% reflected in the typeset document.  \beginlinemode is called before 
%%% the arguments to \title, etc, are parsed, and \endmode is called
%%% when switching from one section to another.  To insure robustness,
%%% \bye is modified to call \endmode.  Once the title areas have been
%%% parsed, \endmode is made back into a benign \par.

\def\beginlinemode{\endmode
  \begingroup\parskip 0pt\obeylines\def\endmode{\par\endgroup}}
\let\endmode=\par
\outer\def\bye{\endmode\vfill\supereject\end}	% slight modification


%%% \title
%%%
%%% Use this to start the paper.

\newif\iftitle \titlefalse
\newbox\thetitle
\def\title{\titletrue\beginlinemode\gettitle}
{\obeylines\gdef\gettitle #1
  {\global\setbox\thetitle=\vbox\bgroup\titlefont\raggedleft%
   #1
   \def\endmode{\egroup\endgroup\strut\vskip 0.5truein%
                \copy\thetitle \titleskip\titlefalse}}}


%%% \authors
%%%
%%% Use this to set the author list. (\author is a synonym)

\newbox\theauthors
\def\authors{\beginlinemode\frenchspacing\getauthors}
\def\author{\authors}
{\obeylines\gdef\getauthors #1
  {\global\setbox\theauthors=\vbox\bgroup\authorfont\raggedleft%
   #1
   \def\endmode{\egroup\endgroup \copy\theauthors \titleskip}}}
\def\authortag#1{\raise3pt\hbox{\rlap{$^{#1}$}}}
  

%%% \emails
%%%
%%% Use this to set the author email address list. (\email is a synonym)

\newbox\theemails
\def\emails{\beginlinemode\getemails}
\def\email{\emails}
{\obeylines\gdef\getemails #1
  {\global\setbox\theemails=\vbox\bgroup\emailfont\raggedleft%
   #1
   \def\endmode{\egroup\endgroup \copy\theemails \titleskip}}}
  

%%% \institution
%%%
%%% Use this to set an institution address.

\newbox\theinstitution
\def\institution{\beginlinemode\getinstitution}
{\obeylines\gdef\getinstitution #1
  {\global\setbox\theinstitution=\vbox\bgroup\institutionfont\raggedleft%
   #1
   \def\endmode{\egroup\endgroup \copy\theinstitution \titleskip}}}
\def\institutiontag#1{~\raise2pt\hbox{$^{#1}$}}

  
%%% \abstract, \introduction, \methods, \results, \summary, \references
%%%
%%% Use these to delimit the major sections.  They take no arguments.

\def\majorsection#1{\endmode\vskip 1cm plus 0.1cm minus 0.1cm\goodbreak%
  {\def\endmode{\par}}			           % make it passive
  \leftline{\majorsectionfont{#1}}\par\nobreak}

\def\abstract{\majorsection{Abstract}}

\def\introduction{\majorsection{Introduction}}

\def\methods{\majorsection{Methods}}

\def\results{\majorsection{Results}}

\def\summary{\majorsection{Summary and Conclusions}}

\def\references{\majorsection{References}\frenchspacing}


%%% \minorsection
%%%
%%% Use this to demark a minor section.  This command takes one paragraph-
%%% delimited argument.  See the LaTeX kludge below!! This function is
%%% really intended only as a helper function, and the proper interface to
%%% it should be made through \section.

\def\minorsection#1\par{\bigbreak\leftline{\sectionfont{#1}}\par\nobreak}


%%% LaTeX kludge for \section
%%%

\ifLaTeX
  \makeatletter
  \def\section{\@ifstar{\majorsection}{\minorsection}}
  \makeatother
\else
  \def\section{\minorsection}
\fi


%%% \CNS, \TeX, and \LaTeX
%%%
%%% These are provided really only for the sample paper.  Note that the \TeX
%%% definition overrides the LaTeX version, restoring it to match Don Knuth's
%%% version in manmac.tex.  The \LaTeX macros work around a bug in Leslie
%%% Lamport's version in latex.ltx.

\def\CNS#1{CNS\lower0.4ex\hbox{*}#1}
\def\TeX{T\kern-.2em\lower.5ex\hbox{E}\kern-.06emX}
\def\LATEX{L\kern-.3em {\setbox0\hbox{T}%
                        \vbox to\ht0{\hbox{\theextrafont A}\vss}}%
            \kern-.15em%
            \TeX}
\def\LaTeX{L\kern-.3em {\setbox0\hbox{T}%
                        \vbox to\ht0{\hbox{\the\scriptfont0 A}\vss}}%
            \kern-.15em%
            \TeX}
\def\LaTeXi{L\kern-.3em {\setbox0\hbox{T}%
                         \vbox to\ht0{\hbox{\the\scriptfont1 A}\vss}}%
            \kern-.25em%
            \TeX}



%%%%%%%%%%%%%%%%%%%%%%%%%%%%%%%%%%%%%%%%%%%%%%%%%%%%%%%%%%%%%%%%%%%%%%%%%%
%%%%%%%%%%%%%%%%%%%%%%%%%%%%%%%%%%%%%%%%%%%%%%%%%%%%%%%%%%%%%%%%%%%%%%%%%%
%%%%%%%%%%%%%%%%%%%%%%%%%%%%%%%%%%%%%%%%%%%%%%%%%%%%%%%%%%%%%%%%%%%%%%%%%%


%%% Sample paper
%%%
%%% The following sample paper was written to exercise these macros and
%%% provide a clean example of what to do and how to use the macros.  This
%%% paper is written for TeX.  To get the same results in LaTeX, there are
%%% three lines that need to be changed, two at the start, and one at the
%%% end; these lines are commented out and marked with an "enable for
%%% LaTeX" message (and the one to be disabled at the end marked "disable
%%% for LaTeX").  

% %%%
% %%% sample.tex
% %%%
% %%% This is a sample paper for the CNS*95 proceedings which describes the 
% %%% usage of the TeX macro set.  It is extensively self-referential, so 
% %%% having a typeset copy to read while perusing the source code would be
% %%% useful. 
% %%%
% %%% pz, 7/95.
% 
% 
% % \documentclass{article}[10pt]         % enable for LaTeX
% \input cns95
% % \begin{document}                      % enable for LaTeX
% 
% \title
% A SET OF \TeX\ AND \LATEX\ MACROS
% FOR FORMATTING \CNS95 PAPERS
% 
% \authors
% John S. Pezaris\authortag{*}
% Richard A. Andersen\authortag{*,\dagger}
% 
% \emails
% pz@caltech.edu
% andersen@vis.caltech.edu
% 
% \institution
% \institutiontag{*}Computation and Neural Systems
% California Institute of Technology, Pasadena, CA 91125, U.S.A.
% 
% \institution
% \institutiontag{\dagger}Division of Biology
% California Institute of Technology, Pasadena, CA 91125, U.S.A.
% 
% \abstract
% 
% A set of macros for formatting papers to the \CNS95 specification using
% \TeX\ and \LaTeX\ is presented.  Each user-level macro is described, using
% this paper, typeset with the macro set, as a running example.  The macros
% are available anonymously as {\tt
% ftp.bbb.caltech.edu:/pub/cns95/cns95.tex}, which also contains the  
% source code to this paper.  The macro set is being placed in the
% public domain so that it may be used freely by authors who wish to
% submit their paper for inclusion in the \CNS95 proceedings.
% 
% 
% \introduction
% 
% The primary contribution of this paper is the description of a set of
% \TeX$^1$ and \LaTeX$^2$ macros which meet the specifications for papers
% submitted for inclusion in the \CNS95 proceedings.$^3$  The macro package is
% quickly introduced in the Methods section, followed by a description of the
% user-level macros, each with usage and a brief example taken from the
% typesetting of this paper.
% 
% The authors' intention is to provide this macro set for public use.  There
% are no restrictions placed on distributing this macro set, however, common
% courtesy would dictate that only the original set be distributed, and that
% any modifications and improvements be sent to the authors for inclusion in
% subsequent versions.  Any bug reports, similarly, should go to the authors,
% and not to the organizers of \CNS95.
% 
% The software is provided free of charge, with the hope that it will be
% useful, but without any warranty, including the implied warranty of
% merchantability or fitness for a particular purpose.  All responsibility for
% any results, whether positive or negative, rests with the reader.
% 
% 
% \section \TeX\ and \LaTeXi
% 
% The same macro set should work with both \TeX\ and \LaTeX, however, as the
% code was written with the \TeX\ user in mind, the \LaTeX\ user will have to
% slightly adapt his or her normal usage.  The macro package is simple enough
% that there should be no major difficulty in doing so.  Throughout this
% paper, the instances where there are specific differences in the two usages
% are pointed out.  Where no difference is noted, the reader may safely assume
% that the described feature works identically in \TeX\ and \LaTeX.  This
% paper was typeset using \TeX; to correctly typeset under \LaTeX, the
% addition of two lines at the start of the document, and the change of a
% third at the end was all that was necessary.
% 
% 
% \methods
% 
% The macro set has been designed to be easy to use, and yet provide an
% attractive layout for the \CNS95 proceedings.  Where possible, the macros
% have been written to produce a source file which mirrors the structure of
% the printed page.  Standard \TeX\ fonts were used throughout, so that the
% macro set {\tt cns95.tex} should be stand-alone.
% 
% A few words about the major parameters.  In \TeX, the magnification is fixed
% at 0 (\LaTeX does not allow the user access to magnification).  Paragraph
% indenting is set to 0~pt, the inter-paragraph spacing to {\tt
% \char92medskipamount}.  The \TeX\ {\tt \char92hsize} and {\tt \char92vsize}
% parameters (\LaTeX\ {\tt \char92textwidth} and {\tt \char92textheight}) are
% set to 4.75~and 7.25~inches, respectively.  In \TeX, the {\tt
% \char92hoffset} parameter is set at 0.875~inches to center the page, and
% {\tt \char92voffset} set at 0.5~inches to start the text 1.5~inches below
% the top of the page; in \LaTeX, the same is effected by {\tt
% \char92topmargin} at 0.5~inches, {\tt \char92headheight} at 0~inches, and
% {\tt \char92headsep} at 0~inches.  Page numbering is turned off.
% 
% 
% %%% \example
% %%%
% %%% A quick little macro to assist in displaying source code.
% 
% \def\example{\bigskip\begingroup\obeylines\advance\leftskip0.5cm
%   \parskip=0pt\par\tt\advance\baselineskip -1pt%
%   \def\\{\char92{}}
%   \def\{{\char123{}}
%   \def\}{\char125{}} 
%   }
% \def\endexample{\par\endgroup\medskip}
% 
% 
% 
% \section Starting the paper
% 
% To get started, a copy of the macro set should be placed the reader's
% current working directory (or in macro library subdirectory, as
% appropriate), and a new file created to write the paper.  If the reader is
% using \TeX, the first line in this new file, after any opening comments,
% should be
% 
% \example
% \\input cns95
% \endexample
% 
% which will load the macros.  If the reader is using \LaTeX, the first few
% lines should be
% 
% \example
% \\documentclass\{article\}[10pt]
% \\input cns95
% \\begin\{document\}
% \endexample
% 
% which will load \LaTeX's article macros and the \CNS95 macro set, and then
% start the document.
% 
% Following this, under both systems, the reader should define the title,
% author list, an optional list of email addresses, the institution list, and
% then start the body of the paper.  The corresponding macros are described
% below.
% 
% 
% \section Defining the title
%  
% To define the title, the {\tt \char92title} macro is used, followed by any
% number of lines of title, in capital letters.  The title for this paper was
% typeset using the \TeX\ code
% 
% \example
% \\title
% A SET OF \\TeX\\ AND \\LaTeX\\ MACROS
% FOR FORMATTING \\CNS95 PAPERS
% \endexample
% 
% Line breaks in the title definition will be reproduced in the typeset
% document. 
% 
% 
% \section Defining the author list
%  
% Next, the author list is defined using a similar mechanism.  Continuing the
% current example, this paper's author list was typeset via
% 
% \example
% \\authors
% John S. Pezaris\\authortag\{*\}
% Richard A. Andersen\char92authortag\{*,\\dagger\}
% \endexample
% 
% As with the title definition, line breaks are reproduced in the typeset
% document.  The {\tt \char92authortag} macro at the end of each line
% is used to denote association with particular institutions, as we will
% see shortly.  If all authors belong to the same institution, the {\tt
% \char92authortag} calls should be omitted.
% 
% 
% \section Defining the optional email address list
%  
% Some authors will wish to publicize their email addresses.  To do this, the
% {\tt \char92emails} macro is used; it is similar to the {\tt
% \char92authors} macro.  The email address list for this paper was typeset
% with
% 
% \example
% \\emails
% pz@caltech.edu
% andersen@vis.caltech.edu
% \endexample
% 
% The teletype font is automatically selected for email addresses, and,
% again, line breaks in the source are reflected in the typeset document.
% Although hopefully, each email address is uniquely associated with one
% author, the {\tt \char92authortag} macro can be used here as well.
% 
% 
% \section Defining the institution list
%  
% The institution list is defined using the {\tt \char92institution} macro,
% once for each institution.  The {\tt \char92institutiontag} macro allows
% symbols to be placed in front of each institute name as footnote
% characters, (ideally) matching the symbols used with {\tt
% \char92authortag} above.  The institution list for this paper consists of
% two departments within Caltech, so two institutions were defined, one for
% each department, and each with a different tag.  In the example below, the
% two definitions are separated by a blank line to increase readability, and
% the institution tags placed {\it before\/} the institution names so that the
% raised symbol appears to the left of the first word in the line.
% 
% \example
% \\institution
% \\institutiontag\{*\}Computation and Neural Systems
% California Institute of Technology, Pasadena, CA 91125, U.S.A.
% \hfill
% \\institution
% \\institutiontag\{\\dagger\}Division of Biology
% California Institute of Technology, Pasadena, CA 91125, U.S.A.
% \endexample
% 
% Again, line breaks in the source code are reflected as line breaks in the
% typeset document (excessively long lines, typical in addresses, will cause
% additional, unplanned, line breaks to be inserted).
% 
% 
% \section The major sections
% 
% The five major section headings as described in the \CNS95 submission
% guidelines are produced via the {\tt \char92abstract}, {\tt
% \char92introduction}, {\tt \char92methods}, {\tt \char92results}, and {\tt
% \char92summary} macros.  An additional sixth major section heading is
% available through the {\tt \char92references} macro.  These six macros take
% no arguments and typeset the major section title in small caps set off
% by clean space.  For example, the beginning of the abstract from this paper
% was typeset as
% 
% \example
% \\abstract
% A set of macros for formatting papers to the \\CNS95
% specification using \\TeX\\ and \\LaTeX\\ is presented.  
% \endexample
% 
% The rest of the major section macros are used in an analogous way.  If you
% want to create your own major section use the {\tt \char92majorsection}
% macro which takes one argument.
% 
% 
% \section Optional minor sections
% 
% For cases where each major section needs to be broken into smaller sections
% (such as this section, entitled ``Optional minor sections''), the {\tt
% \char92section} macro provides just the right typography.  Unlike the major
% section macros which take no arguments, the {\tt \char92section} macro takes one
% argument which runs through to the end of the paragraph.  The argument is
% typeset in italics, and a small break is provided before and after.
% Continuing our example, the first few lines of this section were typeset
% with the source
% 
% \example
% \\section Optional minor sections
% \hfill
% For cases where each major section needs to be broken into
% smaller sections (such as this section, entitled ``Optional
% minor sections''), the \{\\tt \\char92section\} macro provides
% just the right typography.  
% \endexample
% 
% The blank line appearing after the {\tt \char92section} usage is
% important: it is used to delimit the argument to the macro.  \LaTeX\ users
% should note that this redefines the native {\tt \char92section} macro, and
% the various {\tt \char92subsection}, {\tt \char92subsubsection}, 
% etc.~macros will not work properly with this format, so should be avoided.
%  
% 
% \results
% 
% The results of this effort are embodied in the production of this sample
% paper.  This was written using plain \TeX\ and the macro file {\tt
% cns95.tex}.  By making only very minor modifications, the same code will
% produce virtually identical results in \LaTeX; the necessary changes are
% restricted to the two additional lines described above in {\it Starting the
% paper}, and modifying the last line to read {\tt
% \char92end\char123document\char125} instead of {\tt \char92bye}.  To
% provide a ready example, the source code to this paper is included as a
% long comment at the end of the macro file.
% 
% 
% \summary
% 
% A macro set for typesetting papers for inclusion in the \CNS95 proceedings
% has been written and made publicly available.  The macro package was
% designed to meet the specifications given in {\it Guidelines for
% Camera-Ready Manuscript Preparation for CNS$\!$\lower0.5ex\hbox{*}95\/}.$^3$
% The macros are available via anonymous ftp from {\tt
% ftp.bbb.caltech.edu:/pub/cns95/cns95.tex}.  Questions and bugs should be
% directed via email to the corresponding author (JSP) at {\tt pz@caltech.edu}.
% 
%  
% \references
%  
% 1. Knuth, D. E., {\it The \TeX Book}, Addison Wesley (1993).
% 
% 2. Lamport, L., {\it \LaTeXi}, Addison Wesley (1994).
% 
% 3. {\it Guidelines for Camera-Ready Manuscript Preparation for
% CNS$\!$\lower0.5ex\hbox{*}95}, available from Judy Macias, Division of
% Biology 216-76, Caltech, Pasadena, CA 91125.
% 
% % \end{document}                % enable for LaTeX
% \bye                          % disable for LaTeX
% 
% %%%
% %%% End

%%%
%%% End of cns95.tex

